\section{FMI11 --- Kellerautomaten}
\label{sec:fmi}

\subsection{Definition von Automaten}

In der Klausur ist es hilfreich, wenn man die formale Definition von Automaten in natürliche Sprache Übersetzen kann. Denn eigentlich steckt da schon alles drin, was man baucht:

Für jeden Automaten brauchen wir Zustände ($Q$, Menge der Zustände), ein Eingabealphabet ($\Sigma$, Groß--Sigma) und Vorschriften darüber, was beim Lesen von Eingaben getan werden soll ($\sigma$, Klein--Sigma). Des weiteren benötigen wir einen Startzustand ($q_0$) und einen bzw. mehrere Endzustände ($F$). 

Somit ist ein Automat A definiert werden als $A=(Q,\Sigma, \sigma, q_0, F)$.

Bei Kellerautomaten benötigen wir zusätzlich das Kelleralphabet ($\Gamma$, Groß--Gamma, die Menge der Zeichen, die in den Keller geschrieben werden dürfen). Da schon im ersten Ausführungsschritt als erstes ein Zeichen vom Keller gelesen wird, ist ein Zeichennotwendig, das bereits zum Start des Automaten im Keller ist ($k_0$). Daher lautet die formale Definition eines Kellerautomaten $A=(Q,\Sigma, \Gamma, \sigma, q_0, k_0, F)$.

\subsection{Definition von $\sigma$}

In Klausuren sind Eingabealphabet $\Sigma$ meist angegeben. In $Q$ sind alle für die Lösung notwendigen Zustände und die Endzustände in $F$ ergeben sich wieder meistens aus der Aufgabenstellung. Die eigentliche Arbeit liegt also in der Definition der Übergänge $\sigma$ zwischen den Zuständen.

Bei $\sigma$ handelt es sich um eine Funktion. Eine genau definierte Eingabe wird auf eine Ausgabe (deterministische Automaten) oder mehrere (nicht deterministische Automaten) Ausgaben abgebildet. Dies geschieht entweder über ein Diagramm oder eine Automatentafel.

In der Automatentafel sind somit alle Komponenten anzugeben, die man für eine solche Zuordnung benötigt.
Für einen Kellerautomat sind dies:

\begin{table}[h]
\centering
\begin{tabular}{|c|c|c||c|c|}
\hline
\multicolumn{3}{|c||}{Parameter} & \multicolumn{2}{c|}{Funktionswert} \\ \hline
Zustand   & Eingabe   & Keller  & Zustand          & Keller          \\ \hline
$q_0$       & a         & $\#$       & $q_0$              & X               \\ \hline
$q_0$       & a         & X       & $q_1$              & $\varepsilon$               \\ \hline
\multicolumn{3}{|c||}{„Wenn…“} & \multicolumn{2}{c|}{„…dann“} \\ \hline
\end{tabular}
\caption{Automatentafel: Kellerautomat (Beispiel)}
\label{tbl:bspkeller}
\end{table}

Wird der Automat gezeichnet, so müssen diese Informationen ebenfalls im Diagramm zu erkennen sein:

\begin{figure}[H]
\centering
\begin{tikzpicture}[->,>=stealth',shorten >=1pt,auto,node distance=2.8cm,
                    semithick]
%  \tikzstyle{every state}=[fill=red,draw=none,text=white]

  \node[initial,state] (A)                    {$q_0$};
  \node[state]         (B) [right of=A] {$q_1$};

  \path (A) edge [loop above] node {a,$\#$,X} (A)
            edge              node {a,X,$\varepsilon$} (B);
\end{tikzpicture}
\caption{Diagramm: Kellerautomat (Beispiel)}
\label{fig:bspkeller}
\end{figure}


\subsection{Musterklausur 2, Komplexaufgabe 2.2}
\label{sec:mk2-2}

Gegeben ist $L=\{ { a }^{ i }{ b }^{ i }{ c }^{ k }|i,k\in { N }\} \subseteq \left\{ a,b,c \right\} *$.
Somit sollen alle Worte akzeptiert werden, die mit einem oder mehr $a$ beginnen, dann gleich viele $b$ enthalten 
und zuletzt auf ein oder mehr $c$ enden.

Durch die Dreiteilung des Wortes bietet sich ein Automat mit mindestens drei Zuständen an. Jeweils einer zum 
Lesen jedes Buchstabens.

Der Automat muss speichern, wie viele $a$ von der Eingabe gelesen wurden, um anschließend die Zahl der $b$ 
kontrollieren zu können. Hierfür schreibt der Automat in Zustand $q_a$ für jedes gelesene $a$ ein $X$ in den Keller. Hierbei ist zu 
beachten, dass in jedem Schritt zunächt ein Zeichen aus dem Keller geholt wird, welches wieder zurückgelegt wird.
Dies ist beim ersten Durchlauf (unterer Pfeil) das Zeichen für den leeren Keller $\#$ 
und bei allen weiteren Durchläufen (oberer Pfeil) das zuvor geschriebene $X$.

\begin{figure}[H]
\centering
\begin{tikzpicture}[->,>=stealth',shorten >=1pt,auto,node distance=2.8cm,
                    semithick]
%  \tikzstyle{every state}=[fill=red,draw=none,text=white]

  \node[initial,state] 		(A)              {$q_a$};
  \node[state]         		(B) [right of=A] {$q_b$};
  \node[state]         		(C) [right of=B] {$q_c$};

  \path 
  		(A)	edge	[loop below]	node {$a, \#, X\#$} (A)
			edge	[loop above]	node {$a, X, XX$} 		   (A)
			edge					node {$b, X, \varepsilon$} (B)
  		(B)	edge	[loop above]	node {$b, X, \varepsilon$} (B)
			edge					node {$c, \#, \#$} (C)
  		(C)	edge	[loop above]	node {$c, \#, \#$} (C)
  ;
\end{tikzpicture}
\caption{Diagramm: Kellerautomat ($a^ib^ic^k$)}
\label{fig:aibick}
\end{figure}

Sobald das erste $b$ gelesen wird, wechselt der Automat in den Zustand $q_b$. Hier werden dann die $X$ aus dem Keller entnommen
und nichts (also das leere Element $\varepsilon$) zurückgeschrieben. Bis zu dem Punkt, an dem der Keller leer ist und als Eingabe ein $c$
gelesen wird.

In Zustand $q_c$ werden dann die verbleibenden $c$ gelesen, der Keller bleibt leer. Da bei leerem Keller und leerer Eingabe
der Automat das Wort akzeptiert, ist kein weiterer Zustand notwendig.

\begin{table}[h]
\centering
\begin{tabular}{|c|c|c||c|c|}
\hline
\multicolumn{3}{|c||}{Parameter} & \multicolumn{2}{c|}{Funktionswert}         \\ \hline
Zustand   	& Eingabe   & Keller  		& Zustand          	& Keller          \\ \hline
$q_a$       & a         & $\#$			& $q_a$             & X\#             \\ \hline
$q_a$       & a         & X       		& $q_a$             & XX              \\ \hline
$q_a$       & b         & X       		& $q_b$     		& $\varepsilon$   \\ \hline
$q_b$       & b         & X       		& $q_b$     		& $\varepsilon$   \\ \hline
$q_b$       & c         & $\#$     		& $q_c$     		& $\#$			  \\ \hline
$q_c$       & c         & $\#$     		& $q_c$     		& $\#$			  \\ \hline
\end{tabular}
\caption{Automatentafel: Kellerautomat ($a^ib^ic^k$)}
\label{tbl:bspkeller}
\end{table}

