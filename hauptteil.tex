\section{Musterklausur 2}
\label{sec:mk2}

\subsection{Komplexaufgabe 2.2}

Gegeben ist $L=\{ { a }^{ i }{ b }^{ i }{ c }^{ k }|i,k\in { N }\} \subseteq \left\{ a,b,c \right\} *$.
Somit sollen alle Worte akzeptiert werden, die mit einem oder mehr $a$ beginnen, dann gleich viele $b$ enthalten 
und zuletzt auf ein oder mehr $c$ enden.

Durch die Dreiteilung des Wortes bietet sich ein Automat mit mindestens drei Zuständen an. Jeweils einer zum 
Lesen jedes Buchstabens.

\begin{figure}[H]
\centering
\begin{tikzpicture}[->,>=stealth',shorten >=1pt,auto,node distance=2.8cm,
                    semithick]
%  \tikzstyle{every state}=[fill=red,draw=none,text=white]

  \node[initial,state] 		(A)              {$q_a$};
  \node[state]         		(B) [right of=A] {$q_b$};
  \node[state]         		(C) [right of=B] {$q_c$};

  \path 
  		(A)	edge	[loop below]	node {$a, \#, \#X$} (A)
			edge	[loop above]	node {$a, X, XX$} 						(A)
			edge					node {$b, X, \varepsilon$} (B)
  		(B)	edge	[loop above]	node {$b, X, \varepsilon$} (B)
			edge					node {$b, \#, \#$} (C)
  		(C)	edge	[loop above]	node {$c, \#, \#$} (C)
  ;
\end{tikzpicture}
\caption{Diagramm: Kellerautomat ($a^ib^ic^k$)}
\label{fig:aibick}
\end{figure}

Der Automat muss speichern, wie viele $a$ von der Eingabe gelesen wurden, um anschließend die Zahl der $b$ 
kontrollieren zu können. Hierfür schreibt der Automat in Zustand $q_0$ für jedes gelesene $a$ ein $X$ in den Keller. Hierbei ist zu 
beachten, dass in jedem Schritt zunächt ein Zeichen aus dem Keller geholt wird, welches wieder zurückgelegt wird.
Dies ist beim ersten Durchlauf (unterer Pfeil) das Zeichen für den leeren Keller $\#$ 
und bei allen weiteren Durchläufen (oberer Pfeil) das zuvor geschriebene $X$.

Sobald das erste $b$ gelesen wird, wechselt der Automat in den Zustand $q_b$. Hier werden dann die $X$ aus dem Keller entnommen
und nichts (also das leere Element $\varepsilon$) zurückgeschrieben. Bis zu dem Punkt, an dem der Keller leer ist und als Eingabe ein $c$
gelesen wird.

In Zustand $q_c$ werden dann die verbleibenden $c$ gelesen, der Keller bleibt leer. Da bei leerem Keller und leerer Eingabe
der Automat das Wort akzeptiert, ist kein weiterer Zustand notwendig.

\begin{table}[h]
\centering
\begin{tabular}{|c|c|c||c|c|}
\hline
\multicolumn{3}{|c||}{Parameter} & \multicolumn{2}{c|}{Funktionswert} \\ \hline
Zustand   	& Eingabe   & Keller  		& Zustand          	& Keller          \\ \hline
$q_a$       & a         & $\#$			& $q_a$             & X               \\ \hline
$q_a$       & a         & X       		& $q_a$             & XX              \\ \hline
$q_a$       & b         & X       		& $q_b$     		& $\varepsilon$   \\ \hline
$q_b$       & b         & X       		& $q_b$     		& $\varepsilon$   \\ \hline
$q_b$       & c         & $\#$     		& $q_c$     		& $\#$			  \\ \hline
$q_c$       & c         & $\#$     		& $q_c$     		& $\#$			  \\ \hline
\end{tabular}
\caption{Automatentafel: Kellerautomat ($a^ib^ic^k$)}
\label{tbl:bspkeller}
\end{table}

